\documentclass[a4paper,11pt]{article}
\usepackage[hidelinks, colorlinks=true, urlcolor=blue, linkcolor=black, citecolor=blue]{hyperref}
\usepackage[english]{babel}
\usepackage{unicode-math}
\usepackage{xunicode}
\usepackage{url}
\usepackage{cite}
\usepackage{graphicx}
\usepackage[justification=centering, labelfont=bf]{caption}
\usepackage{float}
\usepackage{pgfgantt}
\usepackage{marvosym}
\usepackage{siunitx}
\usepackage{multirow}
\usepackage[nottoc,numbib]{tocbibind}
\usepackage{indentfirst}
\setlength{\parindent}{24pt}
\usepackage{fontspec}

\defaultfontfeatures{Scale=MatchLowercase}
\setmainfont[Ligatures=TeX,
BoldFont=texgyrepagella-bold.otf,
BoldItalicFont=texgyrepagella-bolditalic.otf,
ItalicFont=texgyrepagella-italic.otf]{texgyrepagella-regular.otf}
\setsansfont[Ligatures=TeX,
BoldFont=lmsans10-bold.otf,
BoldItalicFont=lmsans10-boldoblique.otf,
ItalicFont=lmsans10-oblique.otf]{lmsans10-regular.otf}
\setmonofont[BoldFont=lmmonolt10-bold.otf,
BoldItalicFont=lmmonolt10-boldoblique.otf,
ItalicFont=lmmono10-italic.otf,
SlantedFont=lmmonoslant10-regular.otf]{lmmono10-regular.otf}
\setmathfont{texgyrepagella-math.otf}
\setmathfont[range={\mathcal,\mathbfcal},StylisticSet=1]{xits-math.otf}
\setlength{\parindent}{24pt}
\setcounter{secnumdepth}{5}
\setcounter{tocdepth}{1}

\usepackage{fancyhdr}
 
\pagestyle{fancy}
\fancyhf{}
\fancyhead[LE,RO]{Héctor Ramón}
\fancyhead[RE,LO]{\leftmark}
\fancyfoot[RE]{Final degree project}
\fancyfoot[LO]{\emph{Web platform for multiplayer programming games}}
\fancyfoot[LE,RO]{\thepage}

\renewcommand{\footrulewidth}{0.5pt}
\setlength{\marginparwidth}{0pt}

\usepackage{titlesec}
\titleformat{\chapter}{\normalfont\bfseries}{\Huge\thechapter}{20pt}{\Huge}

\setcounter{secnumdepth}{5}
\setcounter{tocdepth}{2}
\begin{document}
\begin{titlepage}
\begin{center}
\textsc{\Large Degree Final Project}
\\[1.5cm]
\rule{\linewidth}{0.5mm}
\\[0.4cm]
{\huge
\bfseries
Web platform for massive multiplayer programming games
\\[0.4cm]
}
\rule{\linewidth}{0.5mm}
\\[0.3cm]
{\bfseries
Monitoring report
}
\\[2.5cm]
\begin{center}
\large
Héctor Ramón Jiménez
\end{center}
Directed by Jordi Petit Silvestre
\vfill
{\large
Facultat d'Informàtica de Barcelona
}
\\[0.5cm]
{\large
\today
}
\end{center}
\end{titlepage}
\clearpage
\tableofcontents
\clearpage
\section{Introduction}
\subsection{Playing games while programming}
In 1961, \textbf{Victor Vyssotsky}, a mathematician and computer scientist working at Bell Labs, had an idea. He devised
a computer game, but not a traditional one where the player inputs the different actions from a controller to play it.
No. He wanted to create a game that you could only play by writing a \textbf{computer program}. And so, along with \textbf{Robert
Morris Sr.} and \textbf{Doug McIlroy}, they created \textbf{Darwin} \cite{darwin}: the first programming game.

A \textbf{programming game} is a computer game where the player does not directly interact with the game. Instead, the
player writes a \textbf{computer program} that plays the game. These \textbf{computer programs} are usually called \textbf{artifficial
intelligences} (\textbf{AI}s) because they try to make intelligent decisions to win the game.

\textbf{Darwin} consisted of two or more small programs, written by the players, that were loaded in memory. The main goal
of the game was to spread copies of your own program and find and kill the copies of other players. The game was only
played for a few weeks before Morris developed an ultimate program, as no-one managed to produce anything that could
defeat it.

Since then, many other programming games have been created \cite{pg}. Some of them are even commercial games, like
\textbf{SpaceChem} \cite{spacechem}.
\subsection{Playing with other people}
With the arrival of the \textbf{Internet} and the \textbf{W}orld \textbf{W}ide \textbf{W}eb, there was nothing stopping people from
developing \textbf{multiplayer programming games}.

A \textbf{multiplayer programming game} is a \textbf{programming game} where multiple players compete with each other to win the
game. Thus, the game becomes a challenge where strategy and programming skills make the difference.

Web platforms have been created that allow players to compete with each other easily. For example, \textbf{Robot Game}
\cite{robotgame} is a website where anyone can upload an \textbf{AI} written in \texttt{Python} and compete with other people.
\subsection{Playing while learning}
Writing \textbf{AI}s can be a really fun and rewarding experience because the game allows the player to see how their
\textbf{algorithms work visually}, while competition between players motivates them to \textbf{learn and improve}.

It is not a surprise, then, that programming games are being used in schools to teach students different programming
techniques. For instance, an \textbf{AI programming challenge} is held every semester in the \textbf{Barcelona School of
Informatics} (\textbf{FIB}) where students enrolled in the subject \textbf{Data Structures and Algorithms} (\textbf{EDA}) \cite{eda}
compete with each other in a multiplayer programming game using the \textbf{Jutge} platform \cite{jutge}.

However, current multiplayer programming games feature \textbf{short matches} with a \textbf{small number of players}. There
is no way to perform a challenge with a considerable amount of students playing at the same time. Therefore, multiple
matches are necessary to decide who wrote the best program.
\subsection{A new era: playing with everyone at the same time}
It is time to take multiplayer programming games to the next level, featuring \textbf{huge worlds}, \textbf{long matches} and
a \textbf{massive amount of players in real-time}. Games where the player will feel attached to the match, where constant
\textbf{evolution} and adaptation matters, where your program plays with everyone at the same time. It is time to create
\textbf{massive multiplayer programming games} (\textbf{MMPG}s).
\section{Main goal}
The main goal of this project is to \textbf{develop a set of libraries and components that will allow to create, manage and
deploy massive multiplayer programming games easily}.

At the same time, this project will enable the \textbf{EDA} \cite{eda} subject at the \textbf{FIB} to use \textbf{MMPG}s to evaluate
its students.
\clearpage
\bibliographystyle{plain}
\bibliography{references}
\end{document}
