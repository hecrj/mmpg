\documentclass[a4paper,11pt]{article}
\usepackage[hidelinks, colorlinks=true, urlcolor=blue, linkcolor=black, citecolor=blue]{hyperref}
\usepackage[english]{babel}
\usepackage{unicode-math}
\usepackage{xunicode}
\usepackage{url}
\usepackage{cite}
\usepackage{graphicx}
\usepackage[justification=centering, labelfont=bf]{caption}
\usepackage{float}
\usepackage{pgfgantt}
\usepackage{marvosym}
\usepackage{siunitx}
\usepackage{multirow}
\usepackage[nottoc,numbib]{tocbibind}
\usepackage{indentfirst}
\setlength{\parindent}{24pt}
\usepackage{fontspec}

\defaultfontfeatures{Scale=MatchLowercase}
\setmainfont[Ligatures=TeX,
BoldFont=texgyrepagella-bold.otf,
BoldItalicFont=texgyrepagella-bolditalic.otf,
ItalicFont=texgyrepagella-italic.otf]{texgyrepagella-regular.otf}
\setsansfont[Ligatures=TeX,
BoldFont=lmsans10-bold.otf,
BoldItalicFont=lmsans10-boldoblique.otf,
ItalicFont=lmsans10-oblique.otf]{lmsans10-regular.otf}
\setmonofont[BoldFont=lmmonolt10-bold.otf,
BoldItalicFont=lmmonolt10-boldoblique.otf,
ItalicFont=lmmono10-italic.otf,
SlantedFont=lmmonoslant10-regular.otf]{lmmono10-regular.otf}
\setmathfont{texgyrepagella-math.otf}
\setmathfont[range={\mathcal,\mathbfcal},StylisticSet=1]{xits-math.otf}
\setlength{\parindent}{24pt}
\setcounter{secnumdepth}{5}
\setcounter{tocdepth}{1}

\usepackage{fancyhdr}
 
\pagestyle{fancy}
\fancyhf{}
\fancyhead[LE,RO]{Héctor Ramón}
\fancyhead[RE,LO]{\leftmark}
\fancyfoot[RE]{Final degree project}
\fancyfoot[LO]{\emph{Web platform for multiplayer programming games}}
\fancyfoot[LE,RO]{\thepage}

\renewcommand{\footrulewidth}{0.5pt}
\setlength{\marginparwidth}{0pt}

\usepackage{titlesec}
\titleformat{\chapter}{\normalfont\bfseries}{\Huge\thechapter}{20pt}{\Huge}

\setcounter{secnumdepth}{5}
\setcounter{tocdepth}{2}
\begin{document}
\begin{titlepage}
\begin{center}
\textsc{\Large Degree Final Project}
\\[1.5cm]
\rule{\linewidth}{0.5mm}
\\[0.4cm]
{\huge
\bfseries
Web platform for massive multiplayer programming games
\\[0.4cm]
}
\rule{\linewidth}{0.5mm}
\\[0.3cm]
{\bfseries
Monitoring report
}
\\[2.5cm]
\begin{center}
\large
Héctor Ramón Jiménez
\end{center}
Directed by Jordi Petit Silvestre
\vfill
{\large
Facultat d'Informàtica de Barcelona
}
\\[0.5cm]
{\large
\today
}
\end{center}
\end{titlepage}
\clearpage
\tableofcontents
\clearpage
\section{Introduction}
\subsection{Playing games while programming}

A \textbf{programming game} is a computer game where the player does not directly interact
  with the game. Instead, the player writes a \textbf{computer program} that plays the game. These \textbf{computer programs}
  are usually called \textbf{artifficial intelligences} (\textbf{AI}s) because they try to make intelligent decisions to win the game.

  Programming games exist since long time ago. The first game of this kind was \textbf{Darwin} \cite{darwin},
  which was invented by Victor Vyssotsky in 1961 and later implemented by Douglas McIlroy at Bell Labs. Since then,
  many programming games have been released \cite{pg}. Some of them are even commercial games, like \textbf{SpaceChem} \cite{spacechem}.
\subsection{Playing with other people}
With the arrival of the \textbf{Internet} and the \textbf{W}orld \textbf{W}ide \textbf{W}eb, there was nothing stopping people from developing
  \textbf{multiplayer programming games}.

  A \textbf{multiplayer programming game} is a \textbf{programming game} where multiple players compete with each other to win the game.
  Thus, the game becomes a challenge where strategy and programming skills make the difference.

  Web platforms have been created that allow players to compete with each other easily.
  For example, \textbf{Robot Game} \cite{robotgame} is a website where anyone can upload an \textbf{AI} written in \texttt{Python}
  and compete with other people.
\subsection{Playing while learning}
Writing \textbf{AI}s can be a really fun and rewarding experience because the game allows the player to see how their algorithms
  work visually,
  while competition between players motivates them to learn and improve.

  It is not a surprise, then, that programming games are being used in schools to teach students different programming techniques.
  For instance, an \textbf{AI programming challenge} is held every
  semester in the \textbf{Barcelona School of Informatics} (\textbf{FIB}) where students enrolled in the subject
  \textbf{Data Structures and Algorithms} (\textbf{EDA}) \cite{eda} compete with each other in a multiplayer programming game
  using the \textbf{Jutge} platform \cite{jutge}.
\subsection{A new era: playing with everyone, anytime}
 However, the \texttt{Jutge} platform only supports programming games where a small
  number of \textbf{AI}s play at the same time. This means that multiple matches are needed to evaluate all the students properly.

  The \textbf{main goal} of this project is to improve the \textbf{Jutge} platform so it can support multiplayer programming games where
  \textbf{all the students} can play \textbf{at the same time} in a match that can last for \textbf{weeks}.
\clearpage
\bibliographystyle{plain}
\bibliography{references}
\end{document}
